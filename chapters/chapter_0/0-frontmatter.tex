\addcontentsline{toc}{chapter}{Abstract}
%\chaptermark{Introduction}
\begin{abstract}
    In this report we investigate the possibility of using a custom-made phase plate to generate a flat potential for the creation of a 2D channel between two reservoirs in a quantum gas transport experiment. We first present a brief introduction to transport experiments with quantum gases together with a discussion on the dipole force and a review on spatial light modulation with phase plates.
    Then, we characterize the custom-made phase plate, with a particular focus on evaluating the flatness of the potential for different incoming beam sizes. The best results have been found for beam diameters close to \SI{5}{mm}, although the specifications required a \SI{6}{mm} beam. Some smaller findings include the development of a procedure to find the focal point and to align the phase plate to the beam. Finally, we discuss the potential implications of the inhomogeneity of the optical potential on the actual transport experiments.

\end{abstract}


\clearpage
\begingroup
%\hypersetup{linkcolor=black}
%\pdfbookmark[chapter]{\contentsname}{toc}
\tableofcontents
%\listoffigures
\endgroup
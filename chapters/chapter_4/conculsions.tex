\chapter{Conclusions}
%In this report we have seen how quantum gases are a versatile system for performing quantum simulation experiments, in particular how they can be used to investigate quantum conduction properties. For these experiments a quasi-2D channel that connects two reservoirs can be generated by a blue-detuned laser beam appropriately shaped. In the current experiment, a $0-\pi$ phase plate is used to generate a TEM$_{01}$-like mode that results in the repulsive potential necessary to create the channel. Using a custom-made phase plate, it should be possible to achieve a uniform channel, as previously investigated by Schmidt in his work \cite{schmidt2021}.

Our tests on the custom-made phase plate have produced interesting and sometimes unexpected results. The central finding of this work is that the parameter to which the shape of the final beam is the most sensible is the incoming beam diameter. In particular, it was found that the optimal beam size is closer to \SI{5}{mm} than to the \SI{6}{mm} indicated by the company. Generally, smaller beams produce flatter potentials. At the same time, for smaller beams, the flat region is shorter, such that the potential resembles the TEM$_{01}$ which is already being used in the experiment. A trade-off between the two has to be found considering the requirements of the experiment.

In the process of reaching these conclusions, some secondary interesting findings were made. One of them is that the focal point can be found looking at the distance between the two peaks along the $z$ direction. At the focus, this distance is minimized. We have also found a convenient way to align the phase plate to the centre of the beam, namely looking at an overexposed image and moving the phase plate minimizing the intensity at the nodal plane. Some other secondary findings include the  possibility of adding a pinhole in the telescope to reduce aberrations and of using an achromatic lens to focus the beam after the phase plate. Both techniques seem to produce a flatter potential and a lower darkness.

At the end of this project, we were not able to find an optimal configuration that could be used in the final experiment. More tests have to be done to carefully pick the optics most suitable for this use. The information in this report should help assemble a good configuration. Moreover, the influence of aberrations should be better investigated, for example through the use of a Schak-Hartmann sensor.